\documentclass{article} % For LaTeX2e
\usepackage{nips13submit_e,times}
\usepackage{hyperref}
\usepackage{url}
\usepackage{graphicx}
\graphicspath{ {images/} }
%\documentstyle[nips13submit_09,times,art10]{article} % For LaTeX 2.09


\title{Clustering Wikipedia Articles}


\author{
Lane Aasen\\
Department of Computer Science\\
University of Washington\\
Seattle, WA 98105\\
\texttt{aaasen@cs.washington.edu}\\
}

% The \author macro works with any number of authors. There are two commands
% used to separate the names and addresses of multiple authors: \And and \AND.
%
% Using \And between authors leaves it to \LaTeX{} to determine where to break
% the lines. Using \AND forces a linebreak at that point. So, if \LaTeX{}
% puts 3 of 4 authors names on the first line, and the last on the second
% line, try using \AND instead of \And before the third author name.

\newcommand{\fix}{\marginpar{FIX}}
\newcommand{\new}{\marginpar{NEW}}

\nipsfinalcopy % Uncomment for camera-ready version

\begin{document}


\maketitle

\begin{abstract}
Clustering Wikipedia articles using unsupervised learning techniques including K-Means, Latent Dirichlet Allocation (LDA), and spectral clustering.
\end{abstract}


\section{Dataset}

The provided dataset contains 15,903 Wikipedia articles in tf-idf format. There are 10,574 unique words in this dataset. Each document is represented as a sparse vector with one dimension for each word.

\section{K-Means Clustering}

For the project milestone, I have implemented K-Means clustering on the provided subset of Wikipedia articles.

\subsection{Choosing K}

\subsubsection{Minimizing Distortion}

Given $K$ clusters $C_{1},C_{2},...,C_{K}$ where each cluster is a set of document vectors and $\mu_{i}$ is the centroid of $C_{i}$, the total distortion is defined as follows:

$$\sum_{i=1}^{K}\sum_{d \in C_{i}} ||d - \mu_{i}||^{2}$$

To minimize the distortion, we could set $K$ equal to the number of documents, but then the clusters would be meaningless. We want to choose a $K$ with low distortion that also results in interpretable clusters. Figure 1 shows a plot of $K$ versus total distortion. When $1 \leq K \leq 16$, adding additional clusters has a large impact on the distortion, but once $K > 16$, adding additional clusters has little impact on the distortion. From this alone, it makes sense to set $K=16$ since it provides a good balance of distortion and interpretability.

\clearpage

\begin{figure}[h]
\begin{center}
\framebox{\includegraphics[scale=0.4]{k_vs_distortion}}
\end{center}
\caption{$K$ versus total distortion for $K \in \{1,2,4,...,256\}$}
\end{figure}

\subsection{K and Cluster Size}

As $K$ increases, the clusters become more sparse. Once $K=256$, over half of the clusters have only one document, and are essentially useless. When $K=16$, the median cluster size is 8.5, and the cluster sizes are as follows:

$$[10061, 3013, 1128, 909, 707, 30, 23, 13, 4, 4, 3, 2, 2, 2, 1, 1]$$

Over half of the clusters are very small, and one of the clusters is too large to be interpretable. This indicates that the data has significant outliers and may lack a structure conducive to clustering.

\begin{figure}[h]
\begin{center}
\framebox{\includegraphics[scale=0.4]{k_vs_cluster_size}}
\end{center}
\caption{$K$ versus minimum, median, and maximum cluster size for $K \in \{1,2,4,...,256\}$ with a $log_{2}$ scale on both axes.}
\end{figure}

\subsection{Exploring Clusters}

Table 1 shows the clusters with at least 10 documents for K-Means clustering with $K=16$. The words in each cluster are the dimensions of the centroid with the largest magnitude. The documents shown are those that are closest to the centroid of the cluster.

Overall, the generated clusters make sense, but there are some points of confusion:

\begin{itemize}
\item The words that make up cluster 0 have little relation to each other. This cluster contains the majority of the documents.
\item Cluster 1 contains churches as well as colleges.
\item Cluster 3 contains documents related to TV shows and sports because both contain the word "season."
\end{itemize}

\section{Next Steps}

\subsection{Recursive K-Means}

One of the largest issues with applying K-Means to this dataset is that it produces clusters with huge size variations. Some clusters contain 10,000 documents, whereas others contain only one. Recursively applying K-means to large clusters could address this problem and even produce a sort of hierarchical clustering.

\subsection{Latent Dirichlet Allocation}

Some of the clusters contain documents that refer to difference meanings of the same word. For example, "season" could refer to a football season or a TV show season. I would like to explore Latent Dirichlet Allocation and whether or not it could help with this situation.







\begin{table}[t]
\caption{K-Means clusters with $K=16$ and at least 10 documents.}
\label{sample-table}
\begin{center}
    \begin{tabular}{ | c | c | c | c |}
    \hline
    \textbf{Cluster} & \textbf{Size} & \textbf{Words} & \textbf{Documents} \\ \hline
    
0 & 10061 & \parbox[t]{2cm}{females \\ station \\ family \\ located \\ north} & \parbox[t]{8cm}{mcgillpainquestionnaire \\ historyofthefamily \\ thetussaudsgroup \\ nadiraactress \\ mansfieldsummithighschool} \\ \hline 
1 & 3013 & \parbox[t]{2cm}{church \\ college \\ students \\ published \\ institute} & \parbox[t]{8cm}{edmondscommunitycollege \\ helderbergcollege \\ oberlincongregationalchurch \\ lundbyoldchurch \\ dioceseoflimerickandkillaloe} \\ \hline 
2 & 1128 & \parbox[t]{2cm}{party \\ served \\ general \\ member \\ senate} & \parbox[t]{8cm}{partyidentification \\ labourfarmerparty \\ democraticalliancesouthafrica \\ liberaldemocratsitaly \\ christiancreditparty} \\ \hline 
3 & 909 & \parbox[t]{2cm}{season \\ club \\ playing \\ seasons \\ player} & \parbox[t]{8cm}{dancingwiththestars \\ davidmccracken \\ gilbertcurgenven \\ bjsamsamericanfootball \\ livingstonewalker} \\ \hline 
4 & 707 & \parbox[t]{2cm}{album \\ released \\ songs \\ records \\ rock} & \parbox[t]{8cm}{thegreatestdaytakethatalbum \\ conflictingemotions \\ primalscream \\ leftbacklp \\ elisamartin} \\ \hline 
5 & 30 & \parbox[t]{2cm}{nba \\ basketball \\ points \\ season \\ seasons} & \parbox[t]{8cm}{kcjones \\ hakeemolajuwon \\ albertkingbasketball \\ ballstatecardinalsmensbasketball \\ 201011southfloridabullsmensbasketballteam} \\ \hline 
6 & 23 & \parbox[t]{2cm}{riots \\ police \\ murder \\ captured \\ robbery} & \parbox[t]{8cm}{sowetouprising \\ 1992losangelesriots \\ nikolaybogolepov \\ josephlamothe \\ jenmi} \\ \hline 
7 & 13 & \parbox[t]{2cm}{congo \\ subtropical \\ republic \\ zambia \\ zimbabwe} & \parbox[t]{8cm}{republicofcabinda \\ brownrumpedbunting \\ copperbeltprovince \\ leptopelisviridis \\ yellowthroatedpetronia} \\ \hline 
 

 
\end{tabular}
\end{center}
\end{table}


\end{document}
